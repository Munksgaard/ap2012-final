\documentclass [10pt,a4paper]{article}
\usepackage[english]{babel}
\usepackage{a4wide}
\usepackage[T1]{fontenc}
\usepackage[utf8x]{inputenc}
\usepackage{amsmath}
\usepackage{amsfonts}
\usepackage{fancyhdr}
\usepackage{ucs}
\usepackage{graphicx}
\usepackage{pdfpages}

\pagestyle{fancy}
\fancyhead[LO]{Philip Munksgaard}
\fancyhead[RO]{Advanced Programming - Final Exam}
\fancyfoot[CO]{\thepage}

\newcommand{\erlang}[1]{\texttt{#1}}

\title{Advanced Programming - Final Exam}
\author{Philip Munksgaard}

\begin{document}
\maketitle

\section{IVars in Erlang}

\subsection{Implementation}

I have implemented IVars as processes. The functions
\erlang{newVanilla/0} and \erlang{newPrincess/1} create new IVars of
the types indicated by their names, by calling a loop function
(\erlang{vanilla\_loop/0} and \erlang{princess\_loop/1} respectively),
that takes care of handling \erlang{put} commands. As soon as a
\erlang{put} command has been received, they go into another loop
(\erlang{vanilla\_loop/2} and \erlang{princess\_loop/2} respectively)
that has the value as an argument \erlang{T} to the
function. \erlang{vanilla\_loop/2} also has a \erlang{Compromised}
argument, which is initially set to false, but if the function
receives a put request, it is set to true.

The argument to the \erlang{put} command for princess IVars are of
course tested against the predicate \erlang{P} before assignment, and
we use \erlang{catch} to make sure exceptions don't interfere with our
functions.

The function \erlang{get} blocks until a value has been \erlang{put}
to that IVar. The loops do so by replying the value \erlang{notset},
whereupon the \erlang{get} function tries again by calling itself. If
the IVar has been set (using \erlang{put}) the loops return
\erlang{\{ok, TheValue\}}.

\subsection{Assesment}

I have written unit tests using EUnit in the pm\_tests module covering
all the cases described in the assignment. I can therefore reasonably
assess that my IVars work as expected and as indicated by my
tests. They probably be more efficient though, for example, instead of
busy waiting in the \erlang{get} function, i could utilize erlangs
message queue in the princess and vanilla loops, and in \erlang{get}
simply wait for a message returning the result of the call.

\end{document}
